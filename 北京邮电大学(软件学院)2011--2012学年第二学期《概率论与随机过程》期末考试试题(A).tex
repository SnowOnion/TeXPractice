% -*- coding: utf-8 -*-
% a4paper - A4纸 11pt -字号 twoside -双面 openany -新章节可在偶数页开始
\documentclass[a4paper,11pt]{article}
\usepackage{geometry}   %设定页边距
\geometry{left=2.5cm,right=2.5cm,top=2.5cm,bottom=2.5cm}

%%%%%%%%%%%%%%% Lee++
\usepackage{CJK}    %中文环境
\usepackage{ulem}   %\uline
\newcommand{\blank}{\uline{\textcolor{white}{a}\ \textcolor{white}{a}\ \textcolor{white}{a}\ \textcolor{white}{a}\ \textcolor{white}{a}\ \textcolor{white}{a}\ \textcolor{white}{a}\ \textcolor{white}{a}\ \textcolor{white}{a}\ \textcolor{white}{a}\ \textcolor{white}{a}}}


% amsmath package, useful for mathematical formulas
\usepackage{amsmath}
% amssymb package, useful for mathematical symbols
\usepackage{amssymb}
% graphicx package, useful for including eps and pdf graphics
% include graphics with the command \includegraphics
\usepackage{graphicx}
% cite package, to clean up citations in the main text. Do not remove.
\usepackage{cite}
\usepackage{color}

\usepackage{color}
\usepackage{placeins}
\usepackage{ulem}
\usepackage{titlesec}
\usepackage{graphicx}
\usepackage{colortbl}
\usepackage{listings}
\usepackage{indentfirst}
\usepackage{fancyhdr}

\begin{document}
\begin{CJK}{GBK}{song}


\begin{center}
北京邮电大学 (软件学院) 2011--2012 学年第二学期\\
{\Large
\textbf{《概率论与随机过程》期末考试试题(A)}
}
\\
TeXify: Lee E-mail: snowonionlee@gmail.com
\\
Pdf和TeX源文件发布在 github.com/SnowOnion/TeXPractice
\end{center}

\section*{一. 填空题~(每小题~3 分,共~45 分)}
\begin{enumerate}
\item 设$A,B$为相互独立的随机事件, $P(A)=0.8, P(B)=0.4$, 则$P(A\bar{B})$=~\blank.

\item 设$A,B$为两个随机事件, 已知$\displaystyle{ P(A)=\frac{1}{2}, P(B)=\frac{1}{3}, P(AB)=\frac{1}{4} }$, 则$P(A\cup B)$=~\blank.

\item 设一批产品中共有10件产品, 其中有2件次品, 现不放回地连续任取6件, 则第5次取出次品的概率为~\blank.

\item 已知随机变量$X$的分布律为
\begin{tabular}{l|llll}
$X$ & -1 & 0 & 1 & 2 \\
\hline
$p_k$ & 0.2 & 0.1 & 0.4 & 0.3
\end{tabular},
设$Y=2\vert X\vert +1$, 则$Y$的分布律为~\blank.



\item 设随机变量$X$的分布函数为$F(X)$, 概率密度为
\[
f(x)=
\left\{
\begin{array}{ll}
2e^{-2x},&\textrm{x $>$ 0,} \\
0,&\textrm{x $\le$ 0.}
\end{array}
\right.
\]
则$F(5)$=~\blank.

\item 二维随机变量$(X,Y)$的概率密度为
\begin{displaymath}
f(x,y)=
\left\{
\begin{array}{ll}
\displaystyle{ \frac{1}{2} }, & \textrm{$0 \le x \le 1, 0 \le y \le 2,$} \\
0, & \textrm{$else$.}
\end{array}
\right.
\end{displaymath}
则$P\{X<Y\}$=~\blank.

\item 设随机变量$X \thicksim U(0,1)$, 则随机变量$Y=2X+1$的概率密度$f(y)$=~\blank.

\item 设随机变量$X \thicksim \pi(2), Y \thicksim B(10,0.5)$, 则$E(2X+4Y-1)$=~\blank.

\item 设离散型随机变量$X$的分布律
\begin{displaymath}
P\{X=k\}=\frac{A}{3^k k!}\quad (k=0,1,2,...)
\end{displaymath}
, 则常数$A$=~\blank.

\item 设随机变量$X,Y$相互独立, 且$X \thicksim N(1,4), Y\thicksim N(4,2)$, 则$2X+4Y+1 \thicksim$~\blank.

\item 设随机变量$X,Y$满足: $D(X)=1, D(Y)=4, D(3X-2Y+1)=13$, 则$\rho_{XY}$=~\blank.

\item 设随机变量$X_1,X_2,...,X_n$独立同分布, 分布函数为$F(x)$, 求随机变量$Z=max\{X_1,X_2,...,X_n\}$的分布函数$F_Z(z)$=~\blank.

\item 设随机过程$X(t)=Yt, Y \thicksim N(5,9)$, 则均值函数为~\blank.

\item 设$\{N(t),t\ge 0\}$服从强度为$\lambda$的泊松过程, 则$P\{N(5)=4,N(7)=6\}$=~\blank, $P\{N(7)=6 | N(5)=4\}$=~\blank.

\end{enumerate}

\newpage

\section*{二. (10分)}

设随机变量$X$具有概率密度
\[
f(x)=
    \left\{
        \begin{array}{ll}
            a\cos x,   &\textrm{$\displaystyle{ |x|<\frac{\pi}{2}, }$} \\
            0,      &\textrm{$else$.}
        \end{array}
    \right.
\]
求: (1) 常数$a$, (2) $\displaystyle{ P\{0<X<\frac{\pi}{4}\} }$, (3) $X$的分布函数.

\section*{三. (10分)}

设二维随机变量$(X,Y)$具有概率密度
\[
f(x,y)=
    \left\{
        \begin{array}{ll}
            kxy,    &\textrm{$0<x<y<1,$} \\
            0,      &\textrm{$else$.}
        \end{array}
    \right.
\]
求: (1) 常数$k$, (2) $P\{X+Y<1\}$, (3) 边缘概率密度$f_X(x), f_Y(y)$.

\section*{四. (10分)}

设随机变量$X,Y$相互独立, 均服从区间$(0,1)$上的均匀分布,求: $Z=X+Y$的概率密度.

\section*{五. (15分)}

已知齐次马氏链$\{X_n,n\ge 0\}$, 状态空间为$I=\{0,1,2\}$, 转移矩阵为
%%只为一场梦 摔碎了山河 重定义array行距 http://bbs.ctex.org/forum.php?mod=viewthread&tid=58032
\renewcommand\arraystretch{1.8}
\[
\mathbf{P}=
\begin{pmatrix}
            \displaystyle{ \frac{1}{2} } & \displaystyle{ \frac{1}{2} } & 0\\
            \displaystyle{ \frac{1}{2} } & 0 & \displaystyle{ \frac{1}{2} }\\
            0 & \displaystyle{ \frac{1}{2} } & \displaystyle{ \frac{1}{2} }
\end{pmatrix}
\]

,初始分布为$\displaystyle{ P_0(0)=\frac{1}{3}, P_1(0)=\frac{1}{3}, P_2(0)=\frac{1}{3} }$. \\
\\
(1) 求二步转移矩阵$P(2)$, (2) 求$P\{X_2=1,X_4=0,X_5=1\}$, (3) 证明遍历性,并求平稳分布.

\section*{六. (10分)}

设$X(t),Y(t),t\ge 0$是相互独立的平稳过程, 验证$Z(t)=X(t)+Y(t)$是否是平稳过程.

\end{CJK}
\end{document}
